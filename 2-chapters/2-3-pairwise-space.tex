\chapter{Projection in the pairwise space}
\label{sec:unchapitre}
\minitoc


\noindent Chapeau introductif
\begin{itemize}
	\item Le calcul d'une métrique implique toujours 2 individus. On va proposer un changement d'espace, un nouvel espace : la représentation par paire.
	\item Le cadre : on suppose que l'on a p métriques.
\end{itemize}

%---------------------------------------------------------------------------
\section{Pairwise embedding}
\begin{itemize}
	\item Changement de l'espace
	\item Normalisation de l'espace des paires
	\item Label des pairwise
\end{itemize}


%---------------------------------------------------------------------------
\section{Interpretation of the pairwise space}
\begin{itemize}
	\item Proximity to the origin (les individus sont identiques)
	\item Proximity of 2 pairwise points in the pairwise space 
	\item Norm in the pairwise space
	\item Representation of combined metric in the pairwise space
\end{itemize}
%%% Local Variables: 
%%% mode: latex
%%% TeX-master: "../roque-phdthesis"
%%% End: 

%---------------------------------------------------------------------------
\section{Pros \& Cons}
\begin{itemize}
	\item perte de la classe initiale des individus. L'information qui nous reste est : les 2 individus sont de la même classe ou sont de classes différentes.
\end{itemize}
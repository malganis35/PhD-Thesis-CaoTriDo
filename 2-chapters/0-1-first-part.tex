\part{Work positionning}

The first part of the manuscript aims to position the work context. Our objective is the comparison and the classification or regression of time series. The first chapter considers time series as static vector data and presents classic machine learning algorithms used to classify them. We note that most of these methods relies on the comparison of objects (time series in our case) through a distance measure. In the second chapter, to cope with the characteristics of time series (amplitude, behavior, frequential spectrum, etc.), we recall some basic metrics used to compare time series. We show that time series may be compared by several modalities and at different granularities. We finally cast that learning an adequate distance based on several modalities and several granularities is a key challenge nowadays to well classify time series using classic machine learning algorithms.


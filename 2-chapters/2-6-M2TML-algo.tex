\chapter{$M^2TML$: implementation}
\label{sec:unchapitre}
\minitoc

\noindent Chapeau introductif :
\begin{itemize}
	\item Quel problème on résout?
	\item Donner les étapes principales de résolution (sous forme de puces). Cela doit rester général, clair et concis.
	\item Développer dans chaque section les puces énumérés précédemment.
\end{itemize}

%---------------------------------------------------------------------------
\section{Projection in the pairwise space}
\begin{itemize}
	\item Projection
	\item Log normalization
\end{itemize}
\noindent \textbf{Pairwise space normalization} \\
This operation is performed to scale the data within the pairwise space and ensure comparable ranges for the $p$ basic metrics $d_h$. In our experiment, we use dissimilarity measures with values in $[0;+\infty[$. Therefore, we propose to Z-normalize their log distributions. \\

%---------------------------------------------------------------------------
\section{M-NN M-diff strategy}
\begin{itemize}
	\item Expliquer les différentes stratégies (k-NN VS All / M-NN VS M-diff / k-NN VS Imposters)
	\item Expliquer pourquoi on va choisir une stratégie M-NN VS M-diff
\end{itemize}


%---------------------------------------------------------------------------
\section{Radius normalization}
\begin{itemize}
	\item Expliquer le problème de la non-homogénéité des radius.
	\item Expliquer comment on résout ce problème par une normalisation des radius de chaque voisinage.
\end{itemize}


%---------------------------------------------------------------------------
\section{Solving the SVM problem}
\begin{itemize}
	\item Expliquer l'apprentissage avec le SVM.
	\item Utilisation de la version L1 du SVM pour avoir une solution sparse.
\end{itemize}


%---------------------------------------------------------------------------
\section{Definition of the dissimilarity measure}
\begin{itemize}
	\item Produit scalaire
	\item Papier PR : norme pondérée x fonction exponentielle
	\item Version Sylvain : norme x fonction exponentielle?
\end{itemize}


%---------------------------------------------------------------------------
\section{Extension to regression problem}
\textit{(To do)}

%---------------------------------------------------------------------------
\section{Extension to multivariate problem}
\textit{(To do)}



%%% Local Variables: 
%%% mode: latex
%%% TeX-master: "../roque-phdthesis"
%%% End: 

\chapter{Time series advanced metrics}
\label{sec:unchapitre}
\minitoc

\noindent Chapeau introductif
\begin{itemize}
	\item Objectif : Trouver une distance, combinaison des distances basiques qui donne une bonne classification $k$-NN sur une base de données.
	\item Pourquoi une distance combinée? Dans le cadre de données réelles, plusieurs modalités peuvent être impliquées (forme, valeur, fréquence), de manière globale ou locale.
	\item Dans le cadre des données réelles, plusieurs composantes/modalités peuvent être impliqués (forme, valeur, fréquence). = attribut (feature) en traitement du signal. Hypothèse : valeur sur une série complète, sur un intervalle ou sur une fenêtre (dans le cadre des métriques à base fréquentielle).
\end{itemize}

%-----------------------------------------------------------------------------
\section{Combined metrics for time series}
\begin{itemize}
	\item Certains travaux dans la littérature propose des combinaisons : linéaire, exponentielle, sigmoïde.
	\item Limites:
	\begin{itemize}
		\item Implique que 2 modalités et au niveau global. Pour intégrer d'autres modalités et à d'autres échelles, il faut changer la formule et ajouter de nouveaux hyper-paramètres à optimiser $\rightarrow$ l'apprentissage de ces paramètres est plus long.
		\item La combinaison est définie a priori
		\item La combinaison est indépendante de la tâche d'analyse.
		\item Pour répondre à ces problèmes, certains auteurs proposent d'apprendre une métrique en vue de la tâche d'analyse considérée (classification, régression, clustering).
	\end{itemize}
\end{itemize}


%-----------------------------------------------------------------------------
\section{Metric Learning: state of the art}
\begin{itemize}
	\item Placer le contexte : travaux réalisés dans le cadre de la classification de données statiques.
	\item Présenter l'intuition du Metric Learning sur la base des travaux de Weinberger.
	\item Donner la terminologie (target, imposter, push, pull)
	\item Objectif : push des imposters et pull des targets
	\item Formalisation du problème (optimisation)
	\item Limites:
	\begin{itemize}
		\item On apprend les poids d'une distance de Mahalanobis
		\item L'apprentissage ne prend pas en compte l'aspect multi-modal dans les données
	\end{itemize}
\end{itemize}

%%% Local Variables: 
%%% mode: latex
%%% TeX-master: "../roque-phdthesis"
%%% End: 

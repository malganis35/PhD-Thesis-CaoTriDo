\begin{vcenterpage}
\noindent\rule[2pt]{\textwidth}{0.5pt}
\\
{\large\textbf{Résumé ---}}
    Lorem ipsum dolor sit amet, consectetur adipiscing elit. Sed non risus. Suspendisse lectus tortor, dignissim sit amet, adipiscing nec, ultricies sed, dolor. Cras elementum ultrices diam. Maecenas ligula massa, varius a, semper congue, euismod non, mi. Proin porttitor, orci nec nonummy molestie, enim est eleifend mi, non fermentum diam nisl sit amet erat. Duis semper. Duis arcu massa, scelerisque vitae, consequat in, pretium a, enim. Pellentesque congue. Ut in risus volutpat libero pharetra tempor. Cras vestibulum bibendum augue. Praesent egestas leo in pede. Praesent blandit odio eu enim. Pellentesque sed dui ut augue blandit sodales. Vestibulum ante ipsum primis in faucibus orci luctus et ultrices posuere cubilia Curae; Aliquam nibh. Mauris ac mauris sed pede pellentesque fermentum. Maecenas adipiscing ante non diam sodales hendrerit. Ut velit mauris, egestas sed, gravida nec, ornare ut, mi. Aenean ut orci vel massa suscipit pulvinar. Nulla sollicitudin. Fusce varius, ligula non tempus aliquam, nunc turpis ullamcorper nibh, in tempus sapien eros vitae ligula. Pellentesque rhoncus nunc et augue. Integer id felis.
\\
\\
{\large\textbf{Mots clés :}}
    Série temporelle, Apprentissage de métrique, $k$-NN, SVM, classification, régression.
\\
\noindent\rule[2pt]{\textwidth}{0.5pt}
%\vspace{0.5cm}
%\vspace{0.1cm}

\noindent\rule[2pt]{\textwidth}{0.5pt}
%\begin{center}
%{\large\textbf{Title in english\\}}
%\end{center}
{\large\textbf{Abstract ---}}
    The definition  of  a metric between time series is inherent to several data analysis and mining tasks, including clustering, classification or forecasting.  Time series  data  present naturally several modalities covering their amplitude, behavior or frequential spectrum, that may be expressed with varying delays and at multiple temporal scales \textemdash exhibited globally or locally.  Combining several modalities at multiple temporal scales to learn a holistic metric is a key challenge for many real temporal data applications.  This paper proposes a Multi-modal and Multi-scale Temporal Metric Learning ({\sc m}$^2${\sc tml}) approach  for maximum margin time series nearest neighbors classification. The solution refers to embedding time series into a dissimilarity space where a pairwise {\sc svm} is used to learn the metric. The {\sc m}$^2${\sc tml}  solution  is proposed for  both linear and non linear contexts. A sparse and interpretable variant of the solution  shows the ability of the learned temporal metric to localize accurately discriminative  modalities as well as their temporal scales. 
    A wide range of 30 public and challenging datasets, encompassing images, traces and {\sc ecg} data, that are  linearly or non linearly separable, are used to show the efficiency and the potential of  {\sc m}$^2${\sc tml} for time series nearest neighbors classification.
\\
\\
{\large\textbf{Keywords:}}
    Time series, Metric Learning, $k$-NN, SVM, classification.
\\
\noindent\rule[2pt]{\textwidth}{0.5pt}
\begin{center}
	Schneider Electric	\\
	Université Grenoble Alpes, LIG\\
	Université Grenoble Alpes, GIPSA-Lab \\
\end{center}
\end{vcenterpage}

%%% Local Variables: 
%%% mode: latex
%%% TeX-master: "../roque-phdthesis"
%%% End: 

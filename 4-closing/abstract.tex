\chapter*{}
\thispagestyle{empty}
%\begin{vcenterpage}
\noindent\rule[2pt]{\textwidth}{0.5pt}
\\
{\large\textbf{Résumé ---}}
	L'apprentissage de métriques temporelles est un processus crucial pour la classification supervisée ou non supervisée de séries temporelles. Les séries temporelles sont naturellement caractérisées par différentes modalités (valeurs, formes, spectres des fréquences...). Ces caractéristiques peuvent être observées avec des délais variables, à différentes échelles et impliquant une partie ou totalité des observations. L'apprentissage de métriques temporelles combinant plusieurs modalités à plusieurs échelles temporelles est un défi au cœur de nombreuses applications émergentes visant la classification et la prédiction de séries temporelles complexes. Cette thèse propose une nouvelle approche {\sc m}$^2${\sc tml} (Multi-modal and Multi-scale Temporal Metric Learning) d'apprentissage de métrique temporelle multi-modale et multi-échelle en vue d'une classification robuste par plus proches voisins. La solution est basée sur la projection de paires de séries dans un espace de dissimilarités, dans lequel un processus d'optimisation à vaste marge est opéré. La solution {\sc m}$^2${\sc tml} est proposée à la fois dans le contexte linéaire et non-linéaire, et est étudiée pour différents types de régularisation. Une variante parcimonieuse et interprétable de la solution montre le potentiel de la métrique temporelle apprise à localiser finement les modalités et échelles discriminantes. L'approche est testée sur une trentaine de bases de données publiques de classes linéairement ou non-linéairement séparables, couvrant entre autres des images, des traces et des {\sc ecg}. Les expériences menées attestent de l'efficacité et du potentiel de la méthode {\sc m}$^2${\sc tml} pour la classification de séries temporelles complexes. 
%\\
%\\
%{\large\textbf{Mots clés :}}
%    Série temporelle, Apprentissage de métrique, $k$-NN, SVM, classification.
\\
\noindent\rule[2pt]{\textwidth}{0.5pt}
%\vspace{0.5cm}
%\vspace{0.1cm}



\newpage
\thispagestyle{empty}

\begin{vcenterpage}
\noindent\rule[2pt]{\textwidth}{0.5pt}
%\begin{center}
%{\large\textbf{Title in english\\}}
%\end{center}
{\large\textbf{Summary ---}}
	The definition of a metric between time series is inherent to several data analysis and mining tasks, including clustering, classification or forecasting. Time series data present naturally several modalities covering their amplitude, behavior or frequential spectrum, that may be expressed with varying delays and at multiple temporal scales —exhibited globally or locally. Combining several modalities at multiple temporal scales to learn a holistic metric is a key challenge for many real temporal data applications. This thesis proposes a Multi-modal and Multi-scale Temporal Metric Learning ({\sc m}$^2${\sc tml}) approach for maximum margin time series nearest neighbors classification. The solution lies in embedding time series into a dissimilarity space where a pairwise {\sc svm} is used to learn the metric. The {\sc m}$^2${\sc tml} solution is proposed for both linear and non linear contexts. A sparse and interpretable variant of the solution shows the ability of the learned temporal metric to localize accurately discriminative modalities as well as their temporal scales. A wide range of 30 public and challenging datasets, encompassing images, traces and {\sc ecg} data, that are linearly or non linearly separable, are used to show the efficiency and the potential of {\sc m}$^2${\sc tml} for time series nearest neighbors classification.
%\\
%\\
%{\large\textbf{Keywords:}}
%    Time series, Metric Learning, $k$-NN, SVM, classification.
\\
\noindent\rule[2pt]{\textwidth}{0.5pt}
\begin{center}
	Schneider Electric	\\
	Université Grenoble Alpes, LIG\\
	Université Grenoble Alpes, GIPSA-Lab \\
\end{center}
\end{vcenterpage}

%%% Local Variables: 
%%% mode: latex
%%% TeX-master: "../roque-phdthesis"
%%% End: 

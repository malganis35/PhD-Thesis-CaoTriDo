\chapter*{Acknowledgements}
%\noindent I would like to thanks:
%\begin{itemize}
%	\item my directors
%	\item my GIPSA collegues
%	\item my AMA collegues
%	\item my Schneider collegues
%	\item my parents
%\end{itemize}


%%% Local Variables: 
%%% mode: latex
%%% TeX-master: "../roque-phdthesis"
%%% End: 

C'est au moment des remerciements que vient la plus grand peur, celle d'oublier quelqu'un ... Pour parer à cette éventualité, je commencerai par remercier toutes les personnes qui m'ont aidé et soutenu pendant ces années.

Je remercie tout d'abord mes directeurs de thèse, Ahlame, Michèle et Sylvain, qui m'ont encadré pendant ces 3 années de thèse, et même avant, dès mon stage de Master 2, pour m'avoir donné les connaissances en Machine Learning, la patience et la rigueur en recherche. Malgré les moments difficiles que nous avons pu rencontrer lors de la thèse, ils ont toujours su m'aider à tenir le bon cap pour arriver aujourd'hui à ce travail d'une excellente qualité. Un grand merci à Ahlame pour m'avoir suivi sur le plan scientifique en me guidant sur la rigueur de recherche et d'écriture; à Sylvain, pour nous avoir partagé ses idées bouillonnantes et sa motivation en tant qu'ingénieur et chercheur; et enfin à Michèle pour son encadrement, son écoute et ses conseils généraux qu'elle a pu me donner pendant ces 3 années.

Je remercie ensuite les membres du jury, Stéphane Canu et Marc Sebban. Je mesure la chance que j'ai de vous avoir comme rapporteurs. Je remercie à vous et aux autres membres, Patrick Gallinari et Gustavo Camps-Valls, pour vos remarques sur le manuscrit, les questions posées et l'intérêt que vous avez porté à ces travaux.

Je remercie également mes collègues de Gipsa, Laetitia, Lyuba, Stefen, Thibault, Jaume, Victor et leurs petit(e)s ami(e)s, Alexandre et Lilia, pour ces années passées avec vous dans le bureau 1131 où on a pu partager des bons moments de thèse et aussi des moments de doute. Votre soutien moral, nos discussions scientifiques et nos sortis Gipsa ont été pour moi un élément que je n'oublierai jamais de cette expérience !

Je n'oublie pas mes collègues de AMA qui cette fois sont nombreux (Saeid, Georgios, Adrian, Bikash, Simon, Irina, Jidong, Fabien et tous les autres) qui ont su m'accueillir alors que je n'ai pas pu être là très souvent. Votre gentillesse et les moments passés lors des séminaires au vert de l'équipe ont été des moments de partage que je garderai en souvenir.

De même, je remercie chacun des permanents des 2 équipes (très nombreux que je ne peux tous citer): Gipsa et AMA pour leur accueil, leurs conseils et leurs retours qu'ils ont pu m'apporter sur ma thèse pendant ces 3 années et qui m'ont permis de construire mon projet de thèse.

A tous mes collègues de Schneider, permanents (Daniel, Franck, Jean Louis, Vincent, Véronique, Henri, Olivier, Rodolph, Alfredo, Laurent, Patrick, Yvon, France, Yassine, Bartosz), doctorants (Peter, Chloé, Benoit, Thibaud), presta (Yvon, Pierre, Nelly, Romain, Sophie, Gregory), stagiaires (Léa, Blaise) et alternants (Matthieu, Thomas, Amadou, Omar) qui ont su me faire découvrir l'environnement Schneider pendant ces 4 ans au sein de l'équipe A4S. Un remerciement tout particulier à Didier, Claude et François pour nous avoir soutenu dans ce projet de thèse et pour nous avoir toujours fait des retours pertinents sur le plan scientifique de la thèse. Et enfin, je n'oublierai jamais le A4S Band avec Matthieu et Peter où on a pu apporter la petite touche musicale de variété internationale dans Schneider. Il y a bien sûr tous les autres collègues de Schneider des autres équipes que je tenais à remercier pour ces années, notamment Benoît, qui a été à l'origine avec Patrice de ce projet.

Je remercie tous les amis qui sont venus de loin (Lyon, Avignon, Marseille, Montpellier, Pau, Paris et de la Suisse) pour le jour de la soutenance. Cela m'a beaucoup touché que vous ayez pu faire le déplacement. Un remerciement également à mes amis de Grenoble pour avoir pu partager ce moment avec vous, et un remerciement tout particulier à ma petite élève Candice et ses parents, Pascale et Claude, qui sont venus voir leur professeur de guitare présenté pour une fois autre chose que de la musique.

De manière générale, je remercie tous mes amis (français, québécois, allemands, vietnamiens, chinois, japonais, péruviens, espagnols), mes colocataires (Lauren, Laurence et Quentin) et ma famille qui ont contribué tout au long de ma thèse à veiller à son bon déroulement et à mes parents en particulier pour leur soutien moral et leur aide qui m'ont donné dans ces derniers jours avec le buffet de thèse et le déménagement.

Enfin, je terminerai ces remerciements à ma chère Louise, celle sans qui rien n'aurait pu être possible et qui m'a soutenu très largement pendant ces 3 ans !

Un grand merci à tous !
\chapter*{Introduction}

\section*{Motivation}
Les travaux de cette thèse s'inscrivent dans le contexte d'une thèse CIFRE avec Schneider Electric et deux laboratoires publics de recherche, le LIG et le GIPSA-lab. Au sein de Schneider Electric, la thèse a eu lieu dans l'équipe Analytics for Solutions (A4S), membre de l'entité Stratégie et Technologie. Parmi les nombreux intérêt de l'équipe, dans le cadre de la modélisation de systèmes (\textit{e.g.}, bâtiments, réseaux de capteurs, Internet des Objets), deux sujets sont au moins étudiés: modélisation à partir des lois physiques (modèles boîtes blanches/grises) et modélisation à partir d'algorithmes d'apprentissage statistique (modèles boîtes noires). avec l'augmentation du nombre de données et de capteurs qui permettent de collecter ces données, il devient de plus en plus difficile de modéliser certains systèmes pour certaines tâches de prédiction, à partir des lois de la physique. Parmi les nombreuses applications dans Schneider Electric, certaines vont impliquer en particulier, des données temporelles, \textit{e.g.}, la prédiction de la consommation dans un bâtiment, le capteur virtuel dans des procédés industriels ou encore la détection de fautes. Plus généralement, Schneider Electric, comme de nombreuses autres entreprises et autres domaines (medecine, marketing, météorologie, etc.) se sont intéressés de plus en plus dans les dernières décennies aux problèmes d'apprentissage (classification, régression, clustering) impliquant des séries temporelles à une ou plusieurs dimensions, à différents échantillonnage, etc. En automatique et en traitement du signal, une série temporelle peut être vue comme la réponse d'un système dynamique. Contrairement aux données statiques, les séries temporelles sont des données en général plus challenge, dans le sens ou l'aspect temporel (\textit{i.e.}, l'ordre d'apparition des observations) est une information clé supplémentaire.

\section*{Positionnement du problème et contributions}
Dans cette thèse, on se focalise sur la classification de séries temporelles monovariées, échantillonées avec une fréquence d'échantillonnage fixe, et de même longueur. Parmi les nombreux algorithmes d'apprentissage qu'il existe, certaines approches (\textit{e.g.}, k-Plus Proches Voisins (k-ppv)) classifie les objets sur la base du concept de voisinage. En général, le concept de 'proche' ou 'loin' entre objets est exprimée au travers d'une mesure de distance. Les séries temporelles peuvent être comparées sur la base de leurs amplitudes comme les données statiques, mais également, sur la base d'autres caractéristiques, appelées modalités, comme leur comportement ou leur contenu en fréquence. De nombreuses métriques pour les séries temporelles ont été proposées comme la distance euclidienne \cite{Ding2008}, la corrélation temporelle \cite{Frambourg2013a} ou la distance à base de Fourier \cite{Sahidullah2012a}. Une revue plus détaillée peut être trouvée dans \cite{Montero2014}. En général, les mesures existantes implique une modalité à l'échelle globale (\textit{i.e.}, impliquant systématiquement l'ensemble des observations). Nous pensons que l'aspect multi-échelle des séries (\textit{i.e.}, impliquer une partie des observations), non présente dans les données statiques, pourrait enrichir la définition des métriques existantes.

Dans ce travail, notre objectif est d'apprendre une métrique combinée multi-modale et multi-échelle pour la classification robuste de séries temporelles par plus proche voisins. Les contributions principales de la thèse sont :

\begin{itemize}
	\item[-] La définition d'un nouvel espace de représentation: l'espace des paires où les paires de séries temporelles sont plongés en un vecteur décrit par un ensemble de métriques temporelles de base.
	\item[-] La définition d'une métrique temporelle de base impliquant une modalité à une échelle spécifique.
	\item[-] l'apprentissage d'une métrique multi-modale et multi-échelle pour une classification à vaste marge de séries temporelles.
	\item[-] la définition du problème général d'apprentissage de métriques combinées comme étant un problème d'apprentissage de métrique dans l'espace des paires.
	\item[-] la proposition d'une architecture basé sur les Support Vector Machine (\textsc{svm}) dans le cadre linéaire et non-linéairement séparable pour définir une métrique combinée qui satisfait les propriétés au moins d'une dissimilarité.
	\item[-] la comparaison de l'approche proposée avec les métriques standards sur un vaste nombre de jeux de données publics.
	\item[-] l'analyse de la méthode proposée pour extraire les caractéristiques discriminatoires impliquées dans la définition de la métrique combinée apprise.
\end{itemize}


\section*{Organisation du manuscrit}
La première partie du manuscrit fait un état de l'art des méthodes existantes en apprentissage statistique et des métriques pour les séries temporelles. Le premier chapitre présente les approches classiques en apprentissage. On rappelle le concept général en apprentissage supervisé et on se focalise sur deux approches : les k-Plus Proches Voisins (k-ppv) et les Support Vector Machine (\textsc{svm}). Dans le second chapitre, on revoit la terminologie de base pour des séries temporelles et on fait la revue de trois catégories de distance qui existent au moins pour les séries temporelles : basée sur les amplitudes, les formes et les fréquences. \\
La seconde partie du manuscrit propose une méthode d'Apprentissage de Métriques Multi-modales et Multi-échelles (\textsc{m$^2$tml}) pour la classification robuste de séries temporelles par plus proches voisins. Dans le troisième chapitre, on rappelle le concept d'apprentissage de métriques pour des données statiques et nous nous focalisons sur une architecture d'apprentissage de métriques pour une classification par plus proches voisins proposées par Weinberger \& Saul \cite{Weinberger2009}. Puis, on présente un nouvel espace de représentation, l'espace des paires, basé sur une description multi-modale et multi-échelle des séries. Puis, nous formalisons le problème général d'optimisation de \textsc{m$^2$tml} en utilisant ce nouvel espace. A partir de la formalisation générale, on décline 3 différentes formalisations. La première et la deuxième proposition utilise différentes régularisations, permettant d'apprendre une métrique combinée linéaire ou non-linéaire. Dans la troisième proposition, on présente une architecture basée sur les \textsc{svm} et une solution pour construire la métrique combinée, dans le contexte linéaire et non-linéaire, et qui satisfait les propriétés d'une mesure de dissimilarité. Finalement, le Chapitre 4 présente les expériences menées sur un vaste nombre de 30 jeux de données publics et discute des résultats obtenus.

% Math macros
\newcommand*{\SET}[1]  {\ensuremath{\mathrm{\mathbf{#1}}}}
\newcommand*{\VEC}[1]  {\ensuremath{\boldsymbol{#1}}}
\newcommand*{\MAT}[1]  {\ensuremath{\boldsymbol{#1}}}
\newcommand*{\OP}[1]  {\ensuremath{\boldsymbol{\mathcal{#1}}}}
\newcommand*{\ESP}[1]  {\ensuremath{ \mathbb{E} \left \{#1 \right \}}}
\newcommand*{\ESPENS}[2]  {\ensuremath{ \mathbb{E}_{#1} \left \{#2 \right \}}}
\newcommand*{\NORM}[1]  {\ensuremath{\left\|#1\right\|}}
\newcommand*{\DPR}[2]  {\ensuremath{\left \langle #1,#2 \right \rangle}}
\newcommand*{\FOURIER}[1]  {\ensuremath{\widehat{#1}}}
\newcommand{\eqdef}{\stackrel{\mathrm{def}}{=}}
\newcommand{\argmax}{\operatornamewithlimits{argmax }}
\newcommand{\argmin}{\operatornamewithlimits{argmin }}
%\newcommand{\argmin}{\arg\!\min}
\newcommand{\diag}{\operatorname{diag}}
\newcommand{\ud}{\, \mathrm{d}}
\newcommand{\vect}{\mathrm{Vect}}
\newcommand{\sinc}{\mathrm{sinc}}
\newcommand{\esp}{\ensuremath{\mathrm{E}}} % Problème pour les short captions
\newcommand{\hilbert}{\ensuremath{\mathcal{H}}}
\newcommand{\supps}{\ensuremath{\tilde{\mathrm{supp}}}}
\newcommand{\supp}{\ensuremath{\mathrm{supp}}}
\newcommand{\sgn}{\mathrm{sgn}}
\newcommand{\intTT}{\int_{-T}^{T}}
\newcommand{\intT}{\int_{-\frac{T}{2}}^{\frac{T}{2}}}
\newcommand{\intinf}{\int_{-\infty}^{+\infty}}
\newcommand{\iintinf}{\iint_{-\infty}^{+\infty}}
\newcommand{\iintrr}{\iint\limits_{\SET{R}^2}}
\newcommand{\intr}{\int\limits_{\R}}
\newcommand{\Sh}{\ensuremath{\boldsymbol{U}}}
\newcommand{\C}{\ensuremath{\mathbf{C}}}
\newcommand{\R}{\ensuremath{\mathbf{R}}}
\newcommand{\Z}{\ensuremath{\mathbf{Z}}}
\newcommand{\N}{\ensuremath{\mathbf{N}}}
\newcommand{\K}{\ensuremath{\mathbf{K}}}
\newcommand{\reel}{\mathcal{R}}
\newcommand{\imag}{\mathcal{I}}
\newcommand{\cmnr}{c_{m,n}^\reel}
\newcommand{\cmni}{c_{m,n}^\imag}
\newcommand{\cnr}{c_{n}^\reel}
\newcommand{\cni}{c_{n}^\imag}
\newcommand{\tproto}{g}
\newcommand{\rproto}{\check{g}}
\newcommand{\Tproto}{G}
\newcommand{\Rproto}{\check{G}}
\newcommand{\Tpoly}{F}
\newcommand{\Rpoly}{\check{F}}
\newcommand{\estim}{\tilde{c}}
\newcommand{\egal}{\bar{c}}
\newcommand{\bb}{b}
\newcommand{\bbf}{z}
\newcommand{\bbr}{\zeta}
\newcommand{\LR}{\mathcal{L}_2(\R)}
\newcommand{\LRR}{\mathcal{L}_2(\R^2)}
\newcommand{\LZ}{\ell_2(\Z)}
\newcommand{\LZZ}{\ell_2(\Z^2)}
\newcommand{\peigne}{\ensuremath{\Psi}}
\newcommand{\avec}{\qquad \text{avec} \qquad}

% Theorems definition
\newtheoremstyle{break}
  {11pt}{11pt}%
  {\itshape}{}%
  {\bfseries}{}%
  {\newline}{}%
\theoremstyle{break}

\newtheorem{definition}{Définition}[chapter]
\newtheorem{theoreme}{Théorème}[chapter]
\newtheorem{remarque}{Remarque}[chapter]
\newtheorem{propriete}{Propriété}[chapter]
\newtheorem{exemple}{Exemple}[chapter]

% Example of background tag
% \AddToShipoutPicture{%
% \begin{tikzpicture}[remember picture,overlay]
%   \node [rotate=60,scale=10,text opacity=0.1] at (current page.center) {Brouillon};
% \end{tikzpicture}}

% Example of header tag
% \AddToShipoutPicture{%
% \tikzstyle{block} = [draw, thick, color=blue, scale=1.5,rectangle, minimum height=3em, minimum width=6em]
% \begin{tikzpicture}[remember picture,overlay]
%   \node [coordinate] at (current page.north) (accroche) {};
%   \node [block, below of=accroche] {Diffusion restreinte};
% \end{tikzpicture}}

% Another example of header tag
% \AddToShipoutPicture{%
% \tikzstyle{block} = [draw, thick, color=red, scale=1.5,rectangle, minimum height=3em, minimum width=6em]
% \begin{tikzpicture}[remember picture,overlay]
%   \node [coordinate] at (current page.north) (accroche) {};
%   \node [block, below of=accroche] {Confidentiel Défense};
% \end{tikzpicture}}

%%% Local Variables: 
%%% mode: latex
%%% TeX-master: "../roque-phdthesis"
%%% End: 

% For conditional control flow in draft mode
\usepackage{ifdraft}

% For larger margins when in draft mode
% \ifdraft{\usepackage[rmargin=3in,marginparwidth=2.75in]{geometry}}{}

% For todo notes in the margins
\usepackage{todonotes}

% For highlighting in MS Word-like comments
\usepackage{soul,color}

% This is needed to ensure the line spacing of the margin notes is correct.
\usepackage{setspace}

% For conditional use of the \comment command based on the options provided.
\usepackage{ifthen}

% BEGIN HIGHLIGHTING: The following code provides a highlighting command, which is 
% used in my custom commenting system below.  If you want more information about 
% this code, see the following URL:
%   http://tex.stackexchange.com/questions/5959/cool-text-highlighting-in-latex
\usepackage{tikz}
\usetikzlibrary{calc}
\usetikzlibrary{decorations.pathmorphing}

\makeatletter

\newcommand{\defhighlighter}[3][]{%
	\tikzset{every highlighter/.style={color=#2, fill opacity=#3, #1}}%
}

\defhighlighter{yellow}{.5}

\newcommand{\highlight@DoHighlight}{
	\fill [ decoration = {random steps, amplitude=1pt, segment length=15pt}
	, outer sep = -15pt, inner sep = 0pt, decorate
	, every highlighter, this highlighter ]
	($(begin highlight)+(0,8pt)$) rectangle ($(end highlight)+(0,-3pt)$) ;
}

\newcommand{\highlight@BeginHighlight}{
	\coordinate (begin highlight) at (0,0) ;
}

\newcommand{\highlight@EndHighlight}{
	\coordinate (end highlight) at (0,0) ;
}

\newdimen\highlight@previous
\newdimen\highlight@current

\DeclareRobustCommand*\highlight[1][]{%
	\tikzset{this highlighter/.style={#1}}%
	\SOUL@setup
	%
	\def\SOUL@preamble{%
		\begin{tikzpicture}[overlay, remember picture]
		\highlight@BeginHighlight
		\highlight@EndHighlight
		\end{tikzpicture}%
	}%
	%
	\def\SOUL@postamble{%
		\begin{tikzpicture}[overlay, remember picture]
		\highlight@EndHighlight
		\highlight@DoHighlight
		\end{tikzpicture}%
	}%
	%
	\def\SOUL@everyhyphen{%
		\discretionary{%
			\SOUL@setkern\SOUL@hyphkern
			\SOUL@sethyphenchar
			\tikz[overlay, remember picture] \highlight@EndHighlight ;%
		}{%
	}{%
	\SOUL@setkern\SOUL@charkern
}%
}%
%
\def\SOUL@everyexhyphen##1{%
	\SOUL@setkern\SOUL@hyphkern
	\hbox{##1}%
	\discretionary{%
		\tikz[overlay, remember picture] \highlight@EndHighlight ;%
	}{%
}{%
\SOUL@setkern\SOUL@charkern
}%
}%
%
\def\SOUL@everysyllable{%
	\begin{tikzpicture}[overlay, remember picture]
	\path let \p0 = (begin highlight), \p1 = (0,0) in \pgfextra
	\global\highlight@previous=\y0
	\global\highlight@current =\y1
	\endpgfextra (0,0) ;
	\ifdim\highlight@current < \highlight@previous
	\highlight@DoHighlight
	\highlight@BeginHighlight
	\fi
	\end{tikzpicture}%
	\the\SOUL@syllable
	\tikz[overlay, remember picture] \highlight@EndHighlight ;%
}%
\SOUL@
}
\makeatother
% END HIGHLIGHTING

% BEGIN MS WORD-STYLE COMMENTS
% This is the main function for providing MS Word-style comments.  There are two 
% ways to use it.  First, you can do the following: 
% 
%     \comment[akm]{This is comment text that will appear in the margin.}{This is 
%         body text that will appear in the main output.}
%
% Second, you can do the following, which will appear only as a note in the margin:
%
%     \comment[akm]{Here's a comment that won't highlight any text.  However, it 
%         should still point to the place in the text where it appears.}
% 
% Third, if you're Aaron, then you can leave out the initials.  Obviously, anyone 
% else reading this should consider altering the command for their own default 
% initials.
%
%     \comment{Here's a margin note using the default initials.}
\newcounter{akmctr}
\newcommand{\comment}[3][akm]{%
	% initials of the author (optional) + note in the margin
	\ifdraft{\refstepcounter{akmctr}%
		{%
			\setstretch{1.0}% line spacing
			\todo[color={red!100!green!33},size=\small,fancyline]{%
				\textbf{Comment [\uppercase{#1} \arabic{akmctr}]:}~#2}
			\ifthenelse{\equal{#3}{}}{}{\highlight[red!100!green!33]{#3}}%
		}%
	}{}%
}
% END MS WORD-STYLE COMMENTS

\newcounter{mycomment}
\newcommand{\mycomment}[2][]{%
	% initials of the author (optional) + note in the margin
	\refstepcounter{mycomment}%
	{%
		\setstretch{0.7}% spacing
		\todo[color={red!100!green!33},size=\small, fancyline]{%
			\textbf{Comment [\uppercase{#1}\themycomment]:}~#2}%
	}}

\newcommand{\myincomment}[2][]{%
	% initials of the author (optional) + note in the margin
	\refstepcounter{mycomment}%
	{%
		\setstretch{0.7}% spacing
		\todo[color={red!100!green!33},size=\small,inline, fancyline]{%
			\textbf{Comment [\uppercase{#1}\themycomment]:}~#2}%
	}}


